\chapter*{Propos liminaires}
\label{forewords}
\pagebookmark[level=0]{forewords}{Propos liminaires}
%\backchapter{Propos liminaires}% To avoid a `frontchapter' definition



\lettrine{P}{roposer une synthèse introductive au numérique} pose l'ambition du présent manuel. Pour ce faire, il s'appuie sur la transcription de trois \textsc{Mooc}\sidenote{\textit{Massive Open Online Course}.} introduits par l'\href{https://www.inria.fr/fr}{\textsc{Inria}} --- Institut national de recherche en informatique et automatique --- et diffusés sur la plateforme \href{https://www.fun-mooc.fr/}{\textsc{Fun-Mooc}} --- France université numérique. 

Les deux premiers \textsc{Mooc} s'adressent plus particulièrement aux enseignants du secondaire qui se destinent à intervenir dans les cours d'informatique des nouveaux programmes : ICN --- « Informatique et création numérique » (2017--2019) --- et SNT --- « Sciences numériques et technologie » (2019-2021). Le troisième \textsc{Mooc}, « \textsc{Python} 3 : des fondamentaux aux concepts avancés du langage » (2017--2021), recouvre un public plus large : tous ceux qui désirent s'initier au langage de programmation \textsc{Python}.

Bien que la qualité intrinsèque\sidenote{Remerciements chaleureux sont naturellement adressés aux auteurs et contributeurs de ces \textsc{Mooc}.} de ces formations se suffise à elle-même, compte tenu qu'elles se complètent, il a semblé intéressant de les regrouper au sein d'une sorte de « document unique ».

Dans la mesure où leur licence de publication le permet, une seconde motivation tient au fait que ce document\sidenote{Sources : \href{https://github.com/ejazzfr/Inria-mooc-handbook}{\faGithub\ Inria-mooc-handbook}.} puisse être remanié par tout un chacun ; bien entendu à titre personnel, mais on pense surtout aux enseignants et formateurs.

Cette première version\sidenote{Version \version.} se résume essentiellement à reprendre les contenus des \textsc{Mooc}, notamment en transcrivant les propos oraux des différents intervenants\sidenote{Les vidéos de la partie SNT ne sont pas reproduites à l'écrit, car ce sont des réalisations didactiques scénarisées et indépendantes ; à visionner pour elles-mêmes.} issus des supports multimédias. En l'état, il manque encore un chapitre sur les implications et applications de l'informatique (bioinformatique, médecine, arts, etc.) et un autre sur l'architecture des ordinateurs et des réseaux. Cependant, le taux de recouvrement n'est pas nul car ces sujets sont également abordés dans d'autres sections du manuel. 

Les apports supplémentaires concernent modestement la partie sur les logiciels libres et celle sur le son et la musique. Par ailleurs les chapitres sont peut-être parfois un peu déséquilibrés en termes de volume d'information. Un meilleur réagencement avec le déplacement de certaines parties en annexe est éventuellement à étudier. Toutefois, le manuel reste exploitable comme tel.

\begin{flushright}
Espérant que le lecteur y trouve satisfaction,\\
Bonne consultation,\\
--- La rédaction
\end{flushright}



\vfill\pagebreak\thispagestyle{empty}





