\chapter*{Guide de lecture}
\label{readingguidebookmark}
\pagebookmark[level=0]{readingguidebookmark}{Guide de lecture}
%\chapter*{Guide de lecture\emDashTitle{}Méthode de travail}
%\backchapter{Guide de lecture}% To avoid a `\frontchapter' definition

\setcounter{sidenote}{0}

\lettrine{L}{e choix effectué pour le rendu de ce document} est celui du format PDF --- \textit{Portable Document Format}. En corrélation avec les outils de composition employés, cela offre une mise en page clairement structurée et la plus « propre » possible pour envisager tout autant une éventuelle impression papier\sidenote{Format actuel des imprimeurs, mais déconseillé si l'on a un tant soit peu de fibre écologique !} qu'une visualisation écran \textit{via} ordinateur ou liseuse numérique.

La perspective est ainsi de proposer un document multi\-fonctionnel, proche du canevas des \textsc{Mooc} offerts sur la plateforme \href{www.fun-mooc.fr}{France université numérique} et accessible hors ligne pour servir de référence, soit comme support personnel d'information, soit à des fins de consultation au sein d'un centre de documentation. 

Compte-tenu de sa licence\sidenote{\textit{Creative Commons} CC-BY-NC.} chacun peut faire évoluer le document selon ses besoins et en extraire\sidenote{Notamment pour un document papier, il est nécessaire de supprimer le visionnage des vidéos.} pour son propre usage tout ou partie, par exemple dans un contexte de formation ou d'enseignement.

%\subsection*{Partis pris rédactionnels}
\subsection*{Orientations et cadre}

La vocation d'un manuel rédigé n'est pas la même qu'une formation en ligne. En tant que transcription écrite de documents oraux (vidéos) et interactifs (Web), le style narratif employé est impersonnel mais le maximum d'interactivité est respecté : navigation, hyperliens et quiz. 

\subsubsection*{Aspects linguistiques et principes rédactionnels}

Le français est une langue relativement riche. Aussi, dans la mesure du possible, la rédaction évite les anglicismes ou propose une traduction littérale des termes anglais usités. Néanmoins, pour un manuel ayant pour sujet l'informatique et plus généralement en tant que document technoscientifique, il serait ridicule d'appliquer cette règle aux mots désormais passés dans le vocabulaire courant ou ne disposant pas de traduction satisfaisante --- spécialement en programmation. % \textsc{Python}.

Ainsi, les anglicismes évitables sont modifiés comme par exemple le célèbre « baser sur\sidenote{En français : « baser à ».} » à la place de « fonder sur » ou « librairie » pour « bibliothèque », mais ceux communément admis sont conservés, notamment en expression technique, comme « implémenter » et «~technologie » au lieu respectivement de « implanter » et « technique~».

En tant que manuel écrit, sa rédaction observe dans l'ensemble les recommandations typographiques\sidenote{Pour les plus soucieux de précision, il est possible de consulter et télécharger le dictionnaire raisonné « \href{http://www.orthotypographie.fr/}{Orthotypographie}~» de Jean-Pierre \textsc{Lacroux}.} de la langue française (cf. « \href{http://jacques-andre.fr/faqtypo/lessons.pdf}{Petites leçons de typographie} » par Jacques \textsc{André} et « \href{http://tex.loria.fr/typographie/saudrais-typo.pdf}{Petit typographe rationnel} » par Eddie \textsc{Saudrais}) : lettres capitales accentuées, ponctuation double avec espaces fines, noms propres en petites capitales, pas d'article commençant un titre, mots d'origine étrangère en italique, retrait de la première ligne d'un paragraphe, pas de mot isolé sur une ligne, emphase en italique mais pas en gras, etc. Cependant, là encore, par souci pratique, certaines consignes ne sont pas honorées, que ce soit les césures --- coupure d'un prénom et d'un nom, césure de certains mot à deux syllabes ou commençant par « con-~» --- ou qu'il s'agisse du rajout d'espaces\sidenote{En français écrit, il n'y a normalement pas d'espace entre les lignes de deux paragraphes consécutifs ; mais cette règle est très souvent déroger pour les manuels techniques. Ce document n'est \textit{a priori} ni un essai, ni un roman... } inter-paragraphes pour aérer le texte.

%On peut également mentionner que les règles de un paragraphe avec un maximum d'environ dix lignes égale une idée développée sont respectées dans la mesure du possible. 

Par ailleurs, si l'anglais n'a cure des répétitions, quant à lui le français les tolère peu car « ça sonne mal ». Un effort est fait pour les éviter mais il va sans dire que ce n'est pas toujours possible ou approprié --- c'est en particulier fort délicat lors de la remise en forme de la contribution d'un auteur ; sans conséquence ou changement de sens ? C'est tout le travail d'un(e) secrétaire\sidenote{Métier ayant tendance à être dévalué.} de rédaction...

\subsubsection*{Partis pris techniques}

Un certain nombre de choix techniques sont conjoncturels ou arbitraires. Le premier d'entre eux est de ne s'appuyer que sur des solutions ouvertes voire libres --- au sens \href{https://www.gnu.org/home.fr.html}{GNU des logiciels libres}. 

Au-delà des aspects culturels, la motivation est que l'ensemble des outils employés et des exemples présentés soient accessibles gratuitement à tout un chacun. Aussi, à la fois pour la rédaction et la consultation du document mais encore pour l'apprentissage de \textsc{Python}, la plateforme de prédilection est donc un système d'exploitation \textsc{Linux} et, plus spécifiquement ici, la distribution « grand public » \textsc{Ubuntu MATE} 20.04 LTS. Cette dernière est suffisamment « légère » pour être installée sur des machines relativement peu puissantes et relève d'un paradigme traditionnel de bureau qui ne peut que ravir les seniors...

La composition du document est effectuée à l'aide de l'outil multi-plateforme \LaTeX\sidenote{\href{https://www.latex-project.org/lppl/}{\textit{\LaTeX{} Project Public License}}.} --- distribution \TeX{}Live{}, moteur \LuaLaTeX{} --- et de multiples extensions\sidenote{\href{https://ctan.org/}{\textit{Comprehensive \TeX{} Archive Network}}.} de style. Quant aux illustrations et images, elles sont élaborées ou retouchées au moyen des extensions graphiques \href{https://ctan.org/pkg/pgf}{\textsc{PGF}/Ti\textit{k}Z} ou des logiciels libres\sidenote{\href{https://www.gnu.org/licenses/gpl-3.0.fr.html}{\textit{GNU General Public License}}.} \href{https://inkscape.org/fr/}{\textsc{Inkscape}} et \href{https://www.gimp.org/fr/}{\textsc{Gimp}}.

Au-delà de la stricte mise en forme, l'intérêt majeur du format PDF en consultation numérique\sidenote{Bien que toujours plus populaires et efficients, les formats \emph{Markdown}, \emph{ePub} voire \emph{Jupyter--Notebook} n'ont pas la même vocation, ni de présentation, ni même de publication. Certes, des outils de conversion existent, mais comportent encore des limitations importantes pour être utilisés facilement --- ils sont réservés à des structures de document et des mises en forme relativement simples et, point d'achoppement majeur pour \emph{ePub}, interdisent l'appel à des fontes \textsc{OpenType} : un des intérêts fondamentaux des moteurs de composition \XeTeX{} et \LuaTeX{}. Pour aller plus loin :
\begin{sideitemize}
\item \href{https://calibre-ebook.com/}{Calibre} ;
\item \href{http://pandoc.org/}{Pandoc}.
\end{sideitemize}} est de pouvoir \emph{gérer les hyperliens}, tout autant que d'inclure ou de \emph{lancer des fichiers multimédias}. De ce fait, ces fonctionnalités sont abondamment exploitées dans ce document et nécessitent un lecteur PDF\sidenote{Pour des raisons de sécurité et certainement commerciales, \textsc{Acrobat Reader}™ n'est plus disponible d'un point de vue fonctionnel et pratique (instabilité) pour \textsc{Linux} au-delà de la version 9.4.1.} tel qu'\textsc{Adobe Reader}™ pour les plateformes propriétaires \textsc{Windows} et \textsc{MacOS}.

Une des seules --- si ce n'est la seule --- lacunes des systèmes d'exploitation fondés sur un environnement \textsc{Gnome}/\textsc{Linux} est, malheureusement encore à ce jour, de ne pas proposer une visionneuse disposant de toutes les fonctionnalités du format PDF, notamment multimédias, à savoir \href{https://wiki.gnome.org/Apps/Evince}{Evince} et \href{https://github.com/mate-desktop/atril}{Atril}. Néanmoins, depuis le printemps 2020, le lecteur de document par défaut de l'environnement \textsc{KDE}/\textsc{Linux} \href{https://okular.kde.org/}{Okular} introduit ces fonctionnalités. Ainsi, la lecture de contenus multimédias dans le corps même du document est également possible sous \textsc{Linux}. %+ OCG %Ce manuel en bénéficie.

En revanche, quelque soit le système d'exploitation et le lecteur PDF énoncé, les couches \textit{Optional Content Groups} (OCG) et \textit{Optional Content Membership Dictionaries} (OCMD) sont désormais implémentées. Ces fonctionnalités du format PDF sont au fondement de l'interactivité des quiz du document (cf. infra).


En outre, en situation conventionnelle d'utilisation, le lecteur multimédia\sidenote{Conseil : quel que soit le système d'exploitation, installer le logiciel \href{https://www.videolan.org/index.fr.html}{VLC} ; la lecture des formats vidéos même les moins orthodoxes est généralement acceptée.} installé par défaut est ouvert et permet, à partir du document, de lancer le visionnage vidéo. Il en est de même d'autres applications comme la consultation de documents externes, PDF ou sites Web.

%de bénéficier de tous les compléments offerts : visionnage vidéo, consultation de documents externes, etc.
%\emph{quizz} et QCM d'auto-évaluation, etc.

La navigation au sein même du document peut au choix s'appréhender à partir des signets de sectionnement du document affichés par le lecteur utilisé (\textsc{Adobe Acrobat Reader}™ ou
%\sideNote{Sous \textsc{Linux}, on peut en particulier citer \textsc{Evince} ou \textsc{Okular} --- en revanche, pour des raisons de sécurité et certainement de \emph{marketing}, \textsc{Acrobat Reader}™ n'est plus disponible d'un point de vue fonctionnel et pratique pour \textsc{Linux} au-delà de la version 9.4.1 ---, voire d'autres outils plus éprouvés tels que \textsc{Xpdfreader}. Certains de ces lecteurs sont également proposés pour plateforme \textsc{Windows}™. À noter que la mise en forme des signets n'est possible que pour \textsc{Acrobat Reader}™ : fonte et couleur.} 
autres), de la table des matières, 
%des listes de figures ou autres documents 
ainsi que des sommaires par chapitre\sidenote{La première page des chapitres et parties proposent un sommaire partiel relatif à leur contenu.} ou partie, mais également au moyen des liens directs proposés dans le corps du document. À cette fin, l'affichage des signets représentant la structure du document est ouverte par défaut --- ainsi nommé le « panneau latéral~».

D'un point de vue pratique, l'ensemble des liens cliquables est révélé par survol de la souris et décelable par \href{https://fr.wikipedia.org/wiki/Liste_de_noms_de_couleur}{code couleur} (cf. infra).
%suivant : \cref{chap:I} --- \qnameref{chap:I}.


\subsubsection*{Aspects formels et mise en page}

La conception de ce manuel suit quelques règles simples de mise en forme. Pour une meilleure lisibilité à l'écran et une approche plus moderne à l'impression, la fonte --- ou police de caractères --- est choisie \href{https://fr.wikipedia.org/wiki/Empattement_(typographie)}{sans empattement} (police dite « bâton ») --- \textit{sans serif} en anglais. La préférence s'est portée sur la famille de fontes \textit{Fira Sans}, développé à l'initiative de la fondation \textsc{Mozilla} dans le cadre du projet \textsc{Firefox OS}, semble-t-il avorté à ce jour.

La famille de fontes \textit{Fira Sans} est une des polices bâton de licence ouverte\sidenote{\href{https://scripts.sil.org/cms/scripts/page.php?site_id=nrsi&id=OFL}{\textit{SIL Open Font Licence}}.} les plus complètes en termes de graisses des caractères (\textit{Hair}, \textit{UltraLight}, \textit{Light}, \textit{Regular}, \textit{Medium}, \textit{SemiBold}, \textit{Bold}, \textit{ExtraBold} et \textit{Heavy}) et de glyphes (petites capitales et symboles mathématiques).

Pour la police à chasse fixe --- \textit{monotype} en anglais ---, utilisée dans les listings ou la transcription de commandes système, la fonte \textit{Fira Mono} possède une chasse trop importante et le dévolu s'est jeté sur la fonte \texttt{Ubuntu mono}. Cela a l'avantage de correspondre aux interfaces graphiques du système d'exploitation \textsc{Ubuntu MATE} pris en exemple comme outil de base.

Dans les us et coutumes de typographie, il est communément admis que la lisibilité optimale d'un texte est obtenue pour des longueurs de ligne d'environ quatre-vingts caractères\sidenote{Cela s'avère être la proposition par défaut de tous les éditeurs de texte.} au maximum. Ce faisant et pour disposer d'un texte suffisamment aéré, le corps des caractères est pris pour avoir des lignes de l'ordre de soixante-cinq caractères. 

Par conséquent, la mise en page offre une marge suffisamment importante pour accueillir la plupart des illustrations et des notes additionnelles sans pour autant perturber le fil du texte principal.

En complément, l'usage veut que l'appel aux couleurs se résume là aussi à l'essentiel pour éviter qu'un effet « sapin de Noël » vienne gêner la lecture. Les codes couleurs du document sont associés à deux couleurs principales et deux couleurs secondaires :
\begin{enumerate}
\item un \textcolor{firstcolor}{bleu électrique} pour les hyperliens et les éléments graphiques d'agencement du manuel --- RGB = (83, 104, 120) ;
\item un \textcolor{secondcolor}{rouge-brun} complémentaire au bleu, mais aussi pour les avertissements et certains encarts --- RGB = (153, 25, 25) ;
\item un \textcolor{thirdcolor}{vert canard} (\textit{teal} en anglais) pour les environnements de solution et encarts d'explication --- RGB = (0, 128, 128) ;
\item une \textcolor{fourthcolor}{palette de marrons} quasi exclusive aux illustrations.
\end{enumerate}

Pour ce qui concerne la coloration syntaxique des listings et la simulation des interpréteurs \textsc{Python}, les codes couleurs sont repris soit du thème « \textit{Radiant-MATE} » de la distribution \textsc{Ubuntu} MATE, soit du module \textsc{Python} \textsc{Pygments}, soit des interpréteurs IDLE et \textsc{IPython} tels que configurés par défaut sur la plateforme (cf. infra).


\subsection*{Structure et contenus}

Les possibilités de navigation au sein du document déjà évoquées sont issues de la structuration même des contenus rédactionnels. Un ensemble de conventions graphiques permet le repérage rapide des différents environnements associés. 

\subsubsection*{Arborescence des supports}

Ce manuel constitue le corps des transcriptions des trois \textsc{Mooc} «~Informatique et création numérique », « Sciences numériques et technologie » et le début de « \textsc{Python} : des fondamentaux aux concepts avancés du langage ». %Peu ou prou il en conserve le chapitrage.

\setcounter{video}{-1}
\begin{marginvideo*}[\label{vid:0.0}Éducation à l'informatique, Gérard \textsc{Berry}.]%
	\movie[width=\marginparwidth,showcontrols]%
		{\includegraphics[width=\marginparwidth]{./Images/Pictograms/film-strip-dark-electric-blue.png}}%
		{./Videos/Chapter00/berry-2019-02-06.mp4}%
	\launchvideo{./Videos/Chapter00/berry-2019-02-06.mp4}
\end{marginvideo*}

En parallèle de ces contenus rédactionnels, les éléments multimédias (essentiellement les vidéos) et les documents externes associés (principalement les fiches de travaux pratiques ou propos complémentaires) forment un tout cohérent et sont repris comme liens internes. Cela suppose au préalable d'installer dans le même répertoire que le manuel les arborescences par chapitre sous les répertoires \directory{./Videos} (sans accent) et \directory{./Documents} --- téléchargement puis décompression des archives respectives.

Les vidéos associées au sujet en cours de consultation sont repérées par un grand pictogramme dans la marge de la page (voir ci-contre). Si la visionneuse PDF employée le permet (\textsc{Adobe Reader}™ sous \textsc{MacOSX} ou \textsc{Windows} et \textsc{Okular} sous Linux), la vidéo peut-être lue dans le document lui-même par simple clic sur l'iconographie. Pour d'autre lecteurs PDF (comme \textsc{Evince} ou \textsc{Atril}), il faut cliquer sur le petit pictogramme \faTv{} pour lancer le lecteur multimédia installé par défaut sur le système.

\subsubsection*{Agencement du document}

La teneur du discours se subdivise par parties et chapitres qui correspondent aux diverses thématiques des \textsc{Mooc} qu'ils transcrivent :
\begin{enumerate}[I.]
\item \textttl{Informatique, création numérique ---} Comme son nom l'indique et en tant qu'introduction, cette partie recouvre les principaux chapitres du \textsc{Mooc} éponyme.
\item \textttl{Fondement de l'informatique ---} Les contenus communs aux Mooc ICN et SNT sont présentés dans cette partie.
\item \textttl{Numérique : culture et pratique ---}  Le propos est d'exposer ici les contenus originaux du \textsc{Mooc} SNT, en explicitant quelques domaines d'application où le numérique est devenu incontournable et en offrant des pistes d'activités à réaliser avec les élèves.
\item \textttl{Codage en langage Python ---} Pour clôturer le corps du document, cette partie est également explicite en apportant le début du tronc commun du \textsc{Mooc} consacré à \textsc{Python}.
%\item \textttl{Champs opérationnels Python ---} Cette partie poursuit l'apprentissage de \textsc{Python} en abordant certains aspects pratiques illustrant les champs d'application du langage.
%\item \textttl{Annexes ---}
%\item \textttl{Addenda ---}
\end{enumerate}

Le document se voulant évolutif, des annexes peuvent bien entendu être apportées sur certains sujets additionnels comme par exemple des données historiques sur les calculateurs et l'informatique, les portes logiques et des éléments d'électronique, des notions de traitement de signal en audionumérique ou en multimédia et des applications avec \textsc{Python}/\textsc{Matplotlib}, etc.


\subsubsection*{Signalétique, pictogrammes et visuels spécifiques}

Pour faciliter la lecture et structurer la présentation des contenus, il est fait appel à des environnements repérés par des éléments graphiques et autres pictogrammes, soit disposés en marge, soit dans le corps du texte.

\pagebreak
\begin{remark}
Ceci est un encart de page pour une remarque additionnelle importante au regard du propos principal.
\end{remark}

Ainsi, ces compléments sont indiquées par l'un des pictogrammes suivant : \textcolor{firstcolor}{\faEye} pour une note indépendante, \textcolor{firstcolor}{\faExclamationTriangle} pour une alerte ou \textcolor{firstcolor}{\faQuestion} dans le cas d'une interrogation.
De même, les notes de rédaction\caution[t]<firstcolor>{%
Ceci est un encart de marge, le plus souvent indiquant une note de la rédaction ou un avertissement spécifique.}{Note de la rédaction}
 et certains avertissements sont exprimés en marge de manière explicite dans leur intitulé (voir ci-contre).

\begin{marker}{Titre optionnel}
De manière similaire, une définition ou une note à retenir peut s’inscrire dans cet environnement particulier (en mode « \textit{Post-It} »).
\end{marker}


\begin{linewidthnote}
Des notes numérotées et de pleine page sont également proposées pour faire état d'un conseil important ou d'une procédure à suivre particulièrement signifiante (cf. apprentissage de \textsc{Python}).
\end{linewidthnote}
\setcounter{linewidthnote}{0}

Les documents internes (voir supra) et externes à ouvrir et/ou à télécharger sont désignés par un pictogramme qui représente leur nature de fichier : \textcolor{secondcolor}{\faFileTextO} texte ou \textsc{LibreOffice} \textsc{Writer}, \textcolor{secondcolor}{\faFilePdfO} PDF, \textcolor{secondcolor}{\faFirefox} page Web, \textcolor{secondcolor}{\faVideoCamera} vidéo et \textcolor{secondcolor}{\faExternalLink} pour un lien externe explicite.

Les exercices et leurs solutions bénéficient d'environnement spécifiques (voir ci-dessous). Les compétences requises sont rapportées en marge par trois petits « voyants » qui distinguent leur niveau de réalisation en passant au vert : basique, intermédiaire et avancé. 

\begin{exercise*}[title={Exercice de style}, before skip=8pt, level=intermediate, points=2 points]
\lipsum[2]
\end{exercise*}

\begin{solution*}[title={Solution non moins de style}, before skip=8pt, level=intermediate, points=2 points]
\lipsum[2]
\end{solution*}


À l’instar des exercices, les questionnaires à choix multiples --- ou «~quiz » --- jouissent d'un environnement particulier associé à une interface interactive. Leur présentation suit plus ou moins celle de la plateforme \textsc{Fun-Mooc}. La fonctionnalité manquante --- pas encore implémentée dans l'extension de style \LaTeX{} usitée --- est relative à la comptabilisation des points et à leur enregistrement. Aussi, il appartient au lecteur de jouer le jeu dans l'autoévaluation de ses connaissances.

De manière traditionnelle en Interface humain-machine (\textsc{Ihm}), lorsqu'une seule réponse est correcte, les propositions sont précédées d'un cercle à cocher (\emph{radio button}) ; en revanche, dans le cas de plusieurs solutions possibles, il s'agit de carrés (\emph{check box}). En outre, après validation des réponses (bouton « Vérifier »), leur explication s'affiche en marge ou en infobulle (bouton « Afficher la réponse »).

Qu'il s'agisse des exercices ou des quiz, un système de points peut --- mais pas obligatoirement --- leur être assignés. Les points sont affichés en haut à droite de leur environnement respectif.

\vspace{6pt}

\begin{quiz*}[title=Autoévaluation, points={1,5 points}]
\vspace{-\baselineskip}
	\begin{mcqdemo*}{4}{1,2,3,5,6,7}{Langage de programmation}
		Quel langage de programmation est présenté dans ce \textsc{Mooc} ?
			\mcqproposal[before skip=4pt]{\textsc{Java}}
			\mcqproposal{\textsc{Perl}}
			\mcqproposal{\textsc{C++}}
			\mcqproposal{\textsc{Python}}
			\mcqproposal{\textsc{Fortran}}
			\mcqproposal{\textsc{Pascal}}
			\mcqproposal[after skip=8pt]{\textsc{Php}}
	\end{mcqdemo*}
	\begin{mcqdemo}{1,2,6}{3,4,5}{Modèle de couleur RGB/RVB}%
	<In optics, one can prove that light is composed of three main wavelengths (or colors), which correspond to red, green and blue (RGB). Thus, a particular color may be defined as a linear combination of these three essential components.>
		Quelles sont les composantes de couleur d'une image RGB/RVB ?
			\mcqproposal[before skip=4pt]{Rouge}
			\mcqproposal{Bleu}
			\mcqproposal{Magenta}
			\mcqproposal{Noir}
			\mcqproposal{Jaune}
			\mcqproposal[after skip=8pt]{Vert}
	\end{mcqdemo}
\end{quiz*}

Comme déjà mentionné, les colorations syntaxiques des codes de programmation sont celles présentes par défaut dans la distribution \textsc{Ubuntu} MATE ou fournies par le module \textsc{Python} \textsc{Pygments}. Le langage de programmation est rappelé en marge des listings : \textcolor{firstcolor}{\xfaC}, \textcolor{firstcolor}{\xfaCplusplus}, \textcolor{firstcolor}{\xfaJavaBold}, \textcolor{firstcolor}{\faJs}, \textcolor{firstcolor}{\xfaPython} et ainsi de suite.

\begin{fullwidth}
\hfill
\begin{codebox}[width=0.45\linewidth, nobeforeafter]{java}
public class HelloWorld
{
	public static void main (string[] args)
	{
		System.out.println("Hello world!");
	}
}
\end{codebox}
\qquad\quad %\hfill
\begin{codebox}[width=0.265\linewidth, nobeforeafter]{python}
print("Hello world!")
\end{codebox}
\end{fullwidth}

À des fins didactiques et pour ne pas troubler inutilement le lecteur néophyte, les fenêtres de terminal \textsc{Linux} (\textit{shell}) et des interpréteurs \textsc{Python} ainsi que l'environnement des \textit{notebooks} (cf. partie \textsc{Python}) sont simulées pour avoir à l'écran peu ou prou le même affichage que dans le manuel. Cela produit les différentes configurations à suivre.

\begin{fullwidth}
\setuser{user}
\begin{ubuntu}
mkdir programming
cd programming §\setuser{user, shdirectory=/programming}§
python §\startconsole§
Python 3.8.2 (default, Apr 27 2020, 15:53:34) 
[GCC 9.3.0] on linux
Type "help", "copyright", "credits" or "license" for more information.
>>> 20§\shmult§30
600
>>> a=10
>>> print(a)
10
>>> exit() §\setuser{user, shdirectory=/programming}§
§\startconsole§
\end{ubuntu}
\begin{ipythonminted}
§\ipythonuserprompt{user}{host}{~/programming}{\$}\textcolor{white}{ipython}§
§\ipythontext{Python 3.8.2 (default, Apr 27 2020, 15:53:34)}§
§\ipythontext{Type `copyright`, `credits` or `license` for more information}§
§\ipythontext{IPython 7.13.0 -- An enhanced Interactive Python. Type `?` for help.}§

§\ipythonpromptin{1}§ 20*30
§\ipythonpromptout{1}§ §\ipythontext{600}§
§\ipythonpromptin{2}§ a=10
§\ipythonpromptin{3}§ print(a)
§\ipythontext{10}§
§\ipythonpromptin{4}§ exit()
§\ipythonuserprompt{user}{host}{~/programming}{\$}§
\end{ipythonminted}
\begin{idleshell}[before skip=2pt, after skip=8pt]
Python 3.8.2 (default, Apr 27 2020, 15:53:34)\par
[GCC 9.3.0] on linux\par
Type "help", "copyright", "credits" or "license()" for more information.
\begin{pyconsole}
20*30
a=10
print(a)
\end{pyconsole}
\end{idleshell}
\begin{nbjupyterin}{1}
20 * 30
\end{nbjupyterin}
\begin{nbjupyterout}{1}
600
\end{nbjupyterout}
\begin{nbjupyterin}{2}
a=10
print(a)
\end{nbjupyterin}
\begin{nbjupyterout}{2}
10
\end{nbjupyterout}
\end{fullwidth}

\vspace{-4pt}

Il est néanmoins à noter qu'une fois présentés tous ces différents environnements de développement, c'est une console IDLE qui est privilégiée, car pleinement intégrée au document au moyen de l'extension \LaTeX{} intitulée \textsc{Python}\TeX{} (cf. \cref{chap:X} \cref{subsub:X.1.3.2}).% --- \cref{code:0.1} ---% (voir ci-dessous).

%\begin{idleconsole}
%\begin{pyconsole}
%20*30
%a=10
%print(a)
%\end{pyconsole}
%\end{idleconsole}


\vfill\pagebreak\thispagestyle{empty}

