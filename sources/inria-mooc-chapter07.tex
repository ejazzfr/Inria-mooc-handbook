\chapter[Architecture des ordinateurs et réseaux]{Architecture\newline des ordinateurs et des réseaux}
\label{chap:VII}

\lettrine{L}{blah blah blah}, reblah reblah reblah reblah reblah reblah. 

%----------
\section[]{}
\label{sec:VII.1}



%\vspace*{-0.5pt}
\subsection[]{}
\label{sub:VII.1.1}



\subsubsection[]{}
\label{subsub:VII.1.1.1}


%\vspace*{-0.5pt}
\subsubsection[]{}
\label{subsub:VII.1.1.2}





%\vspace*{-0.5pt}
\subsection[]{}
\label{sub:VII.1.2}










%----------
\section[]{}
\label{sec:VII.2}



%\vspace*{-0.5pt}
\subsection[]{}
\label{sub:VII.2.1}



\subsubsection[]{}
\label{subsub:VII.2.1.1}

\overparagraph{}


\overparagraph{}

%\subsubsection[Technologies]{Technologies}
%\label{subsub:VIII.2.1.2}







\subsubsection[]{}
\label{subsub:VII.2.1.3}


\overparagraph{}


\overparagraph{}




\subsection[]{}
\label{sub:VII.2.2}















%----------
\section[]{}
\label{sec:VII.3}



%\vspace*{-0.5pt}
\subsection[]{}
\label{sub:VII.3.1}


\subsubsection[]{}
\label{subsub:VII.3.1.1}


\overparagraph{}



\overparagraph{}



\overparagraph{}



\subsubsection[]{}
\label{subsub:VII.3.1.2}


\overparagraph{}



\overparagraph{}














%\vspace*{-0.5pt}
\subsection[]{}
\label{sub:VII.3.2}








%----------
\section[]{}
\label{sec:VII.4}



%\vspace*{-0.5pt}
\subsection[]{}
\label{sub:VII.4.1}


\subsubsection[]{}
\label{subsub:VII.4.1.1}

\overparagraph{}



\overparagraph{}




\subsubsection[]{}
\label{subsub:VII.4.1.2}



\overparagraph{}



\overparagraph{}



\overparagraph{}









\subsection[]{}
\label{sub:VII.4.2}






























%----------
\section[Que faire de ces ressources ? Quiz]{Que faire de ces ressources ? Autoévaluation}
\label{sec:VII.5}

Le questionnaire\caution[t]<firstcolor>{%
La présentation des quiz du document\linebreak suit plus ou moins celle de la platefor\-me \textsc{Fun-Mooc}. La fonctionnalité manquante --- pas encore implémentée dans l'extension de style \LaTeX{} usitée --- est relative à la comptabilisation des points et à leur enregistrement. Aussi, il appartient au lecteur de jouer le jeu dans l'auto\-évaluation de ses connaissances.}{Note de la rédaction}
à choix multiple%
\parnote{De manière traditionnelle en \textsc{Ihm}, lorsqu'une seule réponse est correcte, les propositions sont précédées d'un cercle à cocher (\emph{radio button}) ; en revanche, dans le cas de plusieurs solutions possibles, il s'agit de carrés (\emph{check box}). En outre, après validation des réponses (« Vérifier »), leur explication s'affiche en marge ou infobulle (« Afficher la réponse »).}
--- QCM --- à suivre clôture le présent chapitre \qnameref{chap:VII} et correspond à chaque sujet abordé.
\parnotes

\vspace{6pt}
